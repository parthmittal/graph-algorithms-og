\section{The simulation}
We describe a way to use the precomputed values to simulate brandes
algorithm for a single vertex.
First, in simulating the forward phase, there are two cases:

\begin{enumerate}[I]
	\item
		$u \in V(G^r)$.
		Then, for all $v \in V(G^r)$, we already know the values
		$\delta_G(u, v)$ and $\sigma_G(u, v)$.

		For all $v \in V(G) \setminus V(G^r)$, we look at the paths
		through $\ttt{left(v)}$ and $\ttt{right(v)}$, from $u$ to $v$.
		Then depending on the relation of their lengths, we compute
		$\sigma_G(u, v)$, and we're almost done.
		
		What about $P_G(u, v)$? We don't compute it explicitly at all,
		instead using the fact that
		$w \in P_G(u, v) \iff \delta_G(u, w) + 1 = \delta_G(u, v)$ and
		$(w, v) \in E(G)$.
	\item
		$u \notin V(G^r)$.
		We first find $\delta_G(u, v)$ and $\sigma_G(u, v)$
		for all $v \in V(G^r)$.
		The idea is that a shortest path to any vertex in $G^r$ must
		pass through either $\ttt{left}(u)$ or $\ttt{right}(u)$.
		Now, iterating $S_{G^r}(\ttt{left}(u))$ and $S_{G^r}(\ttt{right}(u))$,
		à la Merge Sort, we can compare the distances of the top element
		of each list, and always pick the smaller one (say $v$) and update
		$\delta_G(u, v)$ and $\sigma_G(u, v)$.

		Now, to compute $\delta_G(u, v)$ and $\sigma_G(u, v)$
		for all $u \in V(G) \setminus V(G^r)$, we do exactly the same as
		the previous case, but with the additional possibility that
		there is a path which uses no vertex of $V(G^r)$ from $u$ to $v$.
\end{enumerate}

Now, to simulate the reverse phase, we need to traverse $v\in V(G)$
in non-increasing order of $\delta_G(u, v)$; we use counting-sort to get
the vertices in this order in $O(V)$
(since the maximum value of $\delta_G$ is $V - 1$).

Note that we now have all the information that Brandes' Algorithm has after
the forward phase starting at $u$ in $G$; we simply run the backward phase
as is (except: when we need the parents of a vertex, we traverse all edges
incident on it).
