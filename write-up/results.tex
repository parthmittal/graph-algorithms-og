\chapter{Results}
We used three data sets to compare our serial implementation with a
serial implementation of Brandes Algorithm (code available
\href{https://github.com/parthmittal/graph-algorithms}{here})

Note that some of the graphs we worked with are actually directed;
we just read them as undirected graphs, and removed any loops and multi-edges.
Also, since we require the input graph to be biconnected, we just picked
the largest biconnected component of the graph.

The edge and vertex counts have been rounded to the nearest thousand,
and the running times are in seconds, and the value reported is CPU
time from \verb|qacct| after job termination.

\begin{table}[H]
	\resizebox{\textwidth}{!}{%
\begin{tabular}{|l|l|l|l|l|}
\hline
\thead{Dataset} & \thead{Vertices (largest BCC)} & \thead{Edges (largest BCC)}
& \thead{Runtime (Brandes)} & \thead{Runtime (our)}\\
\hline
	gnutella         & 63K (34K)    & 148K (119K) & 368 & 299\\
\hline
	soc-epinions1    & 76K (36K)    & 509K (365K) & 553 & 565\\
\hline
\end{tabular}
	}
\end{table}

There were plans to run on a few other (larger) datasets, but due to technical
issues \footnote{an amazingly long debugging session that went on till 2 hours
before I submitted this report}
these plans failed to materialize.

Much more data on runnning time, over many more datasets will be added
to the final presentation.

For checking validity of testdata generated (including a very stringent input
format) and comparing the output using a floating point comparator,
the tool \href{https://github.com/MikeMirzayanov/testlib}{testlib}
was invaluable.
Note that an absolute or relative error of $10^{-4}$ was accepted.
