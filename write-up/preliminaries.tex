\chapter{Preliminaries}
\label{prelim}

\section{Notation}
Unless otherwise noted, the graphs we deal with are undirected, unweighted,
simple, and 2-vertex-connected.
The vertex set of a graph $G$ is $V(G)$, the edge set is $E(G)$.

$\sigma_G(s, t)$ is the number of shortest paths between $s$ and $t$ in $G$,
and $\sigma_G(s, t, v)$ is the number of shortest paths between $s$ and $t$,
which pass through $v$.

$\delta_G(s, t)$ is the length of a shortest path between $s$ and $t$
in the graph $G$.

In several places, notation has been abused for convenience;
when the context makes it clear $S$ has been used to mean $\card{S}$,
and also when the graph we are talking about is clear, $V$ and $E$ are
used to mean that graph's vertex and edge-set respectively.

\section{Ear Decomposition}
An ear of a graph is either a simple path or a simple cycle.

An \emph{open ear decomposition} of a graph $G$ is a partition of $E(G)$ into
a sequence of ears $P_0, P_1, \dots P_k$, such that
$P_0$ is a cycle, and for each $i > 0$, $P_i$ is a simple path,
with both of its endpoints in $P_0 \cup P_1 \cup\dots\cup P_{i - 1}$, and
its internal vertices not in any previous ear.

An \emph{ear decomposition} drops the requirement that each $P_i$ for $i > 0$
be a path.

It was shown in \cite{whitney32} that an open ear decomposition exists iff
$G$ is 2-vertex-biconnected (and this is the reason the current implementation
focuses on biconnected components).

We use the ear decomposition extensively;
it suffices to use a decent serial algorithm to compute it, for example the
one in \cite{schmidt13}, which runs in $O(V + E)$.

\section{Betweenness Centrality}
The \emph{betweenness centrality} of a vertex $v$ is defined as
\[ \sum_{s, t \in V(G)\setminus \{v\}}
	\frac{\sigma_G(s, t, v)}{\sigma_G(s, t)} \]

It should be easy to see a naive algorithm to compute betweenness centrality;
note that if $v$ is on at least one shortest path from $s$ to $t$,
then $\sigma_G(s, t, v) = \sigma_G(s, v) \times \sigma_G(v, t)$.
\begin{algorithm}
\caption{Naive betweenness centrality}
\label{bwc-naive}
\begin{algorithmic}
\Function{bwc-naive}{$G$}
	\State $B[v] \gets 0, \forall v \in V(G)$
	\For{$s \in V(G)$}
		\State Find $\sigma_G(s, t)$, and $\delta_G(s, t), \forall t \in V(G)$
		\Comment{Can do this using a BFS with source $s$}
		\For{$v \in V(G)\setminus \{s\}$}
			\For{$t \in V(G)\setminus \{v\}$}
				\If{$\delta_G(s, t) = \delta_G(s, v) + \delta_G(v, t)$}
					\State 
					$B[v] \gets B[v] + \frac{\sigma_G(s, t, v)}{\sigma_G(s, t)}$
				\EndIf
			\EndFor
		\EndFor
	\EndFor
	\State \Return B
\EndFunction
\end{algorithmic}
\end{algorithm}
Unfortunately, Algorithm \ref{bwc-naive} runs in $O(V^3)$; fortunately, it is
not hard to do better.
