\subsection{Saving memory}
A vigilant reader might notice that our algorithm actually uses
$O(V(G^r)^2)$ space, which is prohibitively large.
In this subsection, we present some tricks to reduce this in practice.

\begin{enumerate}[I]
\item
	We don't need $\delta_{G^r}(s, v)$, etc for some sources $s \in V(G^r)$.
	Note that we need these values to simulate Brandes Algorithm for all $u$
	such that either $\ttt{left}(u)$ or $\ttt{right}(u)$ is $s$.
	If there are no such vertices, we don't need to store the information
	for $s$ at all.
\item
	Building on the previous point, we can get rid of $\delta_{G^r}(s, v)$, etc.
	for the source $s$ as soon as we're done with simulating Brandes Algorithm
	for all $u$ which have $s$ as $\ttt{left}(u)$ or $\ttt{right}(u)$.
	The idea is then to compute a level order of $V(G^r)$ after picking an
	arbitrary source and treating all the edges of $G^r$ as undirected.
	Then at any point we only need to keep the information from two adjacent
	levels of nodes in memory.
	While it is quite easy to construct graphs where all the nodes are on two
	adjacent levels, it turns out in practice this leads to dramatic
	improvements.
\end{enumerate}
