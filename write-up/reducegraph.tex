\subsection{Computing the reduced graph}
The main idea is that we can ``collapse'' all edges which touch at least
one vertex of degree $2$.
For example, consider two vertices of degree at least $3$, such that there
is a path connecting them where every internal vertex of the path has
degree $2$. Then we can replace the entire path by a single edge with weight
equal to the number of edges on it.
The ear decomposition is a very useful tool for book-keeping and constructing
$G^r$.

The reduced graph is a weighted undirected multi-graph (why?), and for each
edge, we store its endpoints $u$ and $v$, and a weight $w$.

For every ear in the ear decomposition, there are two cases.
\begin{enumerate}[I]
	\item
		The endpoints of the ear have degree~$\geq 3$. Then we traverse the
		vertices of the ear along the edges of the ear. 
		Define for every vertex $v$: $\ttt{next}(v)$ as the first vertex
		after~$v$ in this traversal which has degree~$\geq 3$.
		Let $w$ be the number of edges between $v$ and $\ttt{next}(v)$ in
		the traversal, $p$ be the vertex just after $v$ in the traversal,
		and $q$ be the vertex just before $\ttt{next}(v)$ in the traversal.

		Then we add the edge $e = \left\{ v, \ttt{next}(v), w \right\}$
		to $G^r$.

		Also, define the functions $\ttt{left}$ and $\ttt{right}$ on
		every vertex in $V(G) \setminus V(G^r)$ as follows:
		For every vertex~$u$ on the path from $v$ to $\ttt{next}(v)$,
		put $\ttt{left}(u) = v$, and $\ttt{right}(u) = \ttt{next}(v)$.
		It is also convenient to define and store $d_L(u)$ as the distance
		from $u$ to $v$, and the $d_R(u)$ as the distance from $u$ to
		$\ttt{next}(v)$.
	\item
		The endpoints of the ear have degree~$2$.  This is the same as
		the previous case, we just assume the endpoints have degree~$3$,
		and proceed as above.
\end{enumerate}

It should be noted that $G^r$ is a multi-graph, since there may be 2 or more
edge-disjoin paths between two vertices of $G^r$ in $G$;
but importantly, the construction above does not use the same edge from $G$
twice.
