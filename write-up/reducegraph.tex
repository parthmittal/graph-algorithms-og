\subsection{Computing the reduced graph}
The main idea is to collapse away the vertices with degree~$2$.
We use the ear decomposition computed in \ref{ED} for bookkeeping.

The reduced graph is an weighted undirected graph, and for each
edge, we store its endpoints $u$ and $v$, a weight $w$, and two additional
vertices $u'$ and $v'$ (more on this below).
Denote the reduced graph by $G^r$.

For every ear in the ear decomposition, there are two cases.
\begin{enumerate}[I]
	\item
		The endpoints of the ear have degree~$\geq 3$. Then we traverse the
		vertices of the ear along the edges of the ear. 
		Define for every vertex $v$: $\ttt{next}(v)$ as the first vertex
		after~$v$ in this traversal which has degree~$\geq 3$.
		Let $w$ be the number of edges between $v$ and $\ttt{next}(v)$ in
		the traversal, $p$ be the vertex just after $v$ in the traversal,
		and $q$ be the vertex just before $\ttt{next}(v)$ in the traversal.

		Then we add the edge $e = \left\{ v, \ttt{next}(v), w, p, q \right\}$
		to $G^r$.
		It is natual to be curious about the reason for this shenanigan
		at this point.
		The main idea is that every time we use the edge~$e$ in a
		shortest path, we will ``open'' it back up,
		and require the immediate predecessor of
		$v$ (which is $p$) when we travel along $e$, and similar for
		$\ttt{next}(v)$ (which is $q$).
		Note that any vertex $u$ between $v$ and $\ttt{next}(v)$
		necessarily has degree~$2$, which means once we know which direction
		we traverse $e$ in to get to $u$, it is simply a matter of looking
		up the corresponding neighbour in the (unique) ear $u$ belongs to.

		Also, define the functions $\ttt{left}$ and $\ttt{right}$ on
		every vertex in $V(G) \setminus V(G^r)$ as follows:
		For every vertex~$u$ on the path from $v$ to $\ttt{next}(v)$,
		put $\ttt{left}(u) = v$, and $\ttt{right}(u) = \ttt{next}(v)$.
		It is also convenient to define and store $d_L(u)$ as the distance
		from $u$ to $v$, and the $d_R(u)$ as the distance from $u$ to
		$\ttt{next}(v)$.
	\item
		The endpoints of the ear have degree~$2$.  This is the same as
		the previous case, we just assume the endpoints have degree~$3$,
		and proceed as above.
\end{enumerate}

It should be noted that $G^r$ is a multi-graph, but importantly, the
construction above does not use the same edge from $G$ twice.
